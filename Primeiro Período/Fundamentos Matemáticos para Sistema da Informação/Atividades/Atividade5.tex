\documentclass[12pt]{article}
\usepackage[brazilian]{babel}
\usepackage[utf8]{inputenc}
\usepackage{graphicx}
\usepackage{placeins}
\usepackage{float}
\usepackage{amsmath}
\usepackage{blkarray}
\usepackage[svgnames]{xcolor}
\usepackage{amsmath}
\usepackage{rotate}              
\usepackage{enumerate, graphics} 
\usepackage{epsfig}
\usepackage{makeidx}
%\usepackage{tikz}
\usepackage[usenames,dvipsnames]{color}
\usepackage{fancyhdr}
\pagestyle{fancy}
\pagestyle{myheadings}
\fancyhf{}
\lhead{}
\rhead{\thepage}
\rfoot{}
\renewcommand{\headrulewidth}{0pt}


\usepackage{array}
\usepackage[justification=centering]{caption}
\usepackage{quoting}
\usepackage[singlelinecheck=false]{caption}
\usepackage{subcaption}
\usepackage{multirow}
\usepackage{multicol}
\usepackage{lmodern}
\usepackage[T1]{fontenc}
\newenvironment{dedication}
{%\clearpage           % we want a new page          %% I commented this
   \thispagestyle{empty}% no header and footer
   \vspace*{\stretch{1}}% some space at the top
   \itshape             % the text is in italics
   \raggedleft          % flush to the right margin
  }
  {\par % end the paragraph
   \vspace{\stretch{3}} % space at bottom is three times that at the top
   \clearpage           % finish off the page
  }
\usepackage{nomencl}
\makenomenclature
\usepackage{bigstrut}
\usepackage{booktabs}
\usepackage{longtable}
\usepackage{tabularx}
\usepackage{tabulary}
\usepackage{tabu}
\usepackage[toc,page]{appendix}
\usepackage{palatino}
\usepackage{amsmath,amsfonts,mathabx}
\usepackage{epigraph}
\usepackage{natbib}
\usepackage{geometry}
\usepackage{verbatim}
\usepackage{pifont}
\usepackage{setspace}
\usepackage{lscape}
\usepackage{pdflscape}
\usepackage{float}
\usepackage{amsmath}
\usepackage{nicefrac}
\usepackage[nottoc,notlof,notlot]{tocbibind}
\usepackage{booktabs,caption,fixltx2e} 
\usepackage[flushleft]{threeparttable}
\usepackage{tocloft}
\usepackage{times}
\renewcommand{\cftsecleader}{\cftdotfill{\cftdotsep}}


\geometry{top=3cm, bottom=2cm, right=2cm, left=3cm}



\begin{document}

\begin{titlepage}
\centering
 \vfill
  \begin{center}


   {\large {UNIVERSIDADE FEDERAL RURAL DE PERNAMBUCO \\
BACHARELADO EM SISTEMAS DA INFORMAÇÃO
}}\\[4cm]

   {\large {\textbf{Aluno} \\
   Ivanildo Batista da Silva Júnior
   }}\\[.5cm]
   
   {\large {\textbf{Professora}\\ Dra. Silvana Bocanegra}}\\[3cm]
   {\large \textbf{Resolução da quinta lista de Fundamentos Matemáticos para Sistemas da Informação I}}\\[10.5cm] 
   
\normalsize {Recife-PE, \today}
\newpage

  \vfill
\end{center}
\end{titlepage}


\newpage

\tableofcontents
\thispagestyle{empty}
\newpage

\newpage
\setcounter{page}{1}

\begin{section}{Questão 1}{Utilizando o \textit{Cocalc} resolva os exercícios que foram modelados na lista da semana passada. Copie e cole
o código gerado em cada questão e também a saída do \textit{Cocalc} com a solução obtida.}\\

\noindent \textcolor{red}{\underline{\underline{PRIMEIRA QUESTÃO DA QUARTA LISTA: Minimizar uma função custo de transporte.}}}\\

\noindent \textcolor{red}{\underline{1º Passo} : Criar um problema de otimização. Aqui estarei usando o parâmetro \textit{maximization} como \textit{False} já que é um problema de minimização de custos.}

\begin{figure}[H]
    \centering
    \includegraphics[scale=1.2]{Figuras/atv5/q11.PNG}
\end{figure}

\noindent \textcolor{red}{\underline{2º Passo} : Declarar as variáveis e associá-las a \textit{p}.Não precisarei inserir entre as restrições que as quantidades de \textit{x} são não negativas, pois em \textit{p} já estarei considerando a condição de não negatividade com o parâmetro \textit{nonnegative} igual a \textit{True}.}\\

\begin{figure}[H]
    \centering
    \includegraphics[scale=1.2]{Figuras/atv5/q12.PNG}
\end{figure}

\noindent \textcolor{red}{\underline{3º Passo}: Adicionando as restrições ao problema}

\begin{figure}[H]
    \centering
    \includegraphics[scale=1]{Figuras/atv5/q13.PNG}
\end{figure}

\noindent \textcolor{red}{\underline{4º Passo}: Definindo a função objetivo, então aqui irei inserir a função de custo que será minimizada.}

\begin{figure}[H]
    \centering
    \includegraphics[scale=1]{Figuras/atv5/q14.PNG}
\end{figure}

\noindent \textcolor{red}{\underline{5º Passo}: Função para resolver o problema.}

\begin{figure}[H]
    \centering
    \includegraphics[scale=1.2]{Figuras/atv5/q15.PNG}
\end{figure}

\noindent \textcolor{red}{\underline{6º Passo}: Gerando os resultados.}

\begin{figure}[H]
    \centering
    \includegraphics[scale=1.2]{Figuras/atv5/q16.PNG}
\end{figure}

\begin{figure}[H]
    \centering
    \includegraphics[scale=1.2]{Figuras/atv5/q1resposta.PNG}
\end{figure}


\noindent \textcolor{red}{\underline{\underline{SEGUNDA QUESTÃO DA QUARTA LISTA: Maximizar uma função lucro de produção.}}}\\

\noindent \textcolor{red}{\underline{1º Passo} : Criar um problema de otimização. Aqui estarei usando o parâmetro \textit{maximization} como \textit{True} já que é um problema de minimização de custos.}

\begin{figure}[H]
    \centering
    \includegraphics[scale=1.2]{Figuras/atv5/q21.PNG}
\end{figure}

\noindent \textcolor{red}{\underline{2º Passo} : Declarar as variáveis e associá-las a \textit{p}.Não precisarei inserir entre as restrições que as quantidades de \textit{x} são não negativas, pois em \textit{p} já estarei considerando a condição de não negatividade com o parâmetro \textit{nonnegative} igual a \textit{True}.}\\

\begin{figure}[H]
    \centering
    \includegraphics[scale=1.2]{Figuras/atv5/q22.PNG}
\end{figure}

\noindent \textcolor{red}{\underline{3º Passo}: Adicionando as restrições ao problema}

\begin{figure}[H]
    \centering
    \includegraphics[scale=1]{Figuras/atv5/q23.PNG}
\end{figure}

\noindent \textcolor{red}{\underline{4º Passo}: Definindo a função objetivo, então aqui irei inserir a função de custo que será minimizada.}

\begin{figure}[H]
    \centering
    \includegraphics[scale=1]{Figuras/atv5/q24.PNG}
\end{figure}

\noindent \textcolor{red}{\underline{5º Passo}: Função para resolver o problema.}

\begin{figure}[H]
    \centering
    \includegraphics[scale=1.2]{Figuras/atv5/q25.PNG}
\end{figure}

\noindent \textcolor{red}{\underline{6º Passo}: Gerando os resultados.}

\begin{figure}[H]
    \centering
    \includegraphics[scale=1.2]{Figuras/atv5/q26.PNG}
\end{figure}

\begin{figure}[H]
    \centering
    \includegraphics[scale=1.2]{Figuras/atv5/q2resposta.PNG}
\end{figure}

\noindent \textcolor{red}{\underline{\underline{TERCEIRA QUESTÃO DA QUARTA LISTA: Minimizar uma função custo para ração de gado.}}}

\begin{figure}[H]
    \centering
    \includegraphics[scale=1]{Figuras/atv5/q31.PNG}
\end{figure}

\noindent \textcolor{red}{Solução para o problema.}

\begin{figure}[H]
    \centering
    \includegraphics[scale=1]{Figuras/atv5/q32.PNG}
\end{figure}

\noindent \textcolor{red}{\underline{\underline{QUARTA QUESTÃO DA QUARTA LISTA: Maximizar taxa de retorno de uma carteira de investimentos.}}}

\begin{figure}[H]
    \centering
    \includegraphics[scale=1]{Figuras/atv5/q41.PNG}
\end{figure}

\noindent \textcolor{red}{Solução para o problema.}

\begin{figure}[H]
    \centering
    \includegraphics[scale=1]{Figuras/atv5/q42.PNG}
\end{figure}

\end{section}
\newpage

\begin{section}{Questão 2}{Resolva os exercícios a seguir usando o método de solução gráfica:}

\begin{itemize}
    \item[a)]
    
$$
\max f(x_1,x_2) = x_1 + 2x_2
$$

sujeito a 

$$
x_1 \leq 2
$$
$$
x_2 \leq 2
$$
$$
x_1 + x_2 \leq 3
$$
$$
x_1 \geq 0; \quad x_2 \geq 0
$$

\noindent \textcolor{red}{RESOLUÇÃO}

\textcolor{red}{Primeiro irei converter as inequações em equações, conforme abaixo}

$$
\textcolor{red}{x_1 = 2}
$$
$$
\textcolor{red}{x_2 = 2}
$$
$$
\textcolor{red}{x_1 + x_2 = 3}
$$

\noindent \textcolor{red}{Já obtemos dois resultados ($x_1 = 2$ e $x_2 = 2$). Para a terceira equação irei encontrar a reta da equação, quando $x_1=0$ e $x_2=0$. Assim sendo, quando $x_1=0$, então $x_2 =3$ e quando $x_2=0$, temos que $x_1 =3$. Os pontos da reta da terceira equação são $A(0,3)$ e $B(3,0)$.}

\begin{figure}[H]
    \centering
    \includegraphics[scale=.8]{Figuras/atv5/questao2a1.PNG}
\end{figure}

\textcolor{red}{Levando em consideração as restrições do problema, a nossa solução se encontra na região na cor amarela, pois os valores devem estar abaixo da reta verde ($x_2 \leq 2$), a esquerda da reta laranja ($x_1 \leq 2$) e abaixo da reta vermelha ($x_1 + x_2 \leq 3$). A condições de não negatividade ($x_1 \geq 0$ e $x_2 \geq 0$), nos garantirão que não usaremos as regiões com valores negativos no gráfico.}\\

\textcolor{red}{Podemos analisar se essa região do gráfico é a que contém nossa solução substituimos o ponto $E(1,1)$ nas restrições e, se para seus valores as restrições se mantiverem verdadeiras, então nossa solução estará dentro da região amarela:}

$$
\textcolor{red}{x_1 \leq 2 \rightarrow 1 \leq 2 \quad \textrm(verdadeira)}
$$
$$
\textcolor{red}{x_2 \leq 2 \rightarrow 1 \leq 2 \quad \textrm(verdadeira)}
$$
$$
\textcolor{red}{x_1 + x_2 \leq 3 \rightarrow 1+1 \leq 3 \rightarrow 2 \leq 3 \quad \textrm(verdadeira)}
$$

\noindent \textcolor{red}{Como todas as restrições se mantiveram verdadeiras, então a área amarela é onde encontra-se a nossa solução.}\\

\noindent \textcolor{red}{Agora irei utilizar múltiplos dos valores de $x_1$ e $x_2$ e analisar como a função se comporta e se seus valores ficam dentro da região amarela. Como múltiplos irei usar 2, 4 e 8 e encontrei os pontos da reta de lucro para cada um desses valores.}

$$
\textcolor{red}{x_1 + 2x_2 = 2 \rightarrow M(0,1) \textrm{ e } N(2,0)}
$$
$$
\textcolor{red}{x_1 + 2x_2 =4 \rightarrow M(0,2) \textrm{ e } N(4,0)}
$$
$$
\textcolor{red}{x_1 + 2x_2 =8 \rightarrow M(0,4) \textrm{ e } N(8,0)}
$$

\noindent \textcolor{red}{Gerando as retas no nosso gráfico temos:}

\begin{figure}[H]
    \centering
    \includegraphics[scale=.7]{Figuras/atv5/questao2a2.PNG}
\end{figure}

\noindent \textcolor{red}{A reta preta refere-se a equação $x_1 + 2x_2 = 2$, a roxa a $x_1 + 2x_2 = 4$ e a rosa a $x_1 + 2x_2 = 8$. Notar que a região abaixo da reta roxa tracejada é maior que a área abaixo da reta preta e que a reta rosa está fora da região de solução. A reta roxa toca na reta de $x_1=2$, portanto já temos o valor de $x_1$ e podemos encontrar o valor de $x_2$ que é 1. Então, 4 é o valor máximo dessa função.}


\item[b)]

$$
\min f(x_1,x_2) = -x_1 -x_2
$$

sujeito a 

$$
-x_1 + x_2 \leq 2
$$
$$
2x_1 - x_2 \leq 6
$$
$$
x_1 \geq 0; \quad x_2 \geq 0
$$

\textcolor{red}{Convertendo as inequações em equações}
$$
\textcolor{red}{-x_1 + x_2= 2}
$$
$$
\textcolor{red}{2x_1 - x_2 = 6}
$$

\noindent \textcolor{red}{Para cada uma das equações iremos encontrar a reta da equação, quando $x_1=0$ e $x_2=0$. Assim sendo, para a primeira, quando $x_1=0$, então $x_2 =2$ e quando $x_2=0$, temos que $x_1 =-2$. Para a segunda equação, quando $x_1=0$, então $x_2 =-6$ e quando $x_2=0$, temos que $x_1 =3$.  Os pontos da reta da primeira equação são $A(0,2)$ e $B(-2,0)$ e para a segunda equação $A(0,-6)$ e $B(3,0)$ e obtemos o gráfico abaixo}

\begin{figure}[H]
    \centering
    \includegraphics[scale=.7]{Figuras/atv5/questao2b2.PNG}
\end{figure}

\noindent \textcolor{red}{A figura acima mostra as duas retas das equações das restrições e a área em amarelo a região que contém o conjunto de valores com a solução do problema. Para confirmarmos se essa região é a que contém nossa solução, vamos substituir os valores do ponto $E(1,1)$ nas inequações e se elas se mantiverem verdadeiras, então nossa solução estará nessa área; caso contrário, fora dela.}

$$
\textcolor{red}{-x_1 + x_2 \leq 2 \rightarrow - 1 + 1 \leq 2 \quad \rightarrow 0 \leq 2 \quad \textrm(verdadeira)}
$$
$$
\textcolor{red}{2x_1 - x_2 \leq 6 \rightarrow 2-1 \leq 6 \quad \rightarrow 1 \leq 6 \quad \textrm(verdadeira)}
$$

\textcolor{red}{Notar que as retas das equações se cruzam, esses pontos são}

$$
\textcolor{red}{-x_1 + x_2 = 2}
$$
$$
\textcolor{red}{x_2 = 2 + x_1}
$$

\textcolor{red}{Substituindo na equação abaixo}

$$
\textcolor{red}{2x_1 - (2 + x_1) = 6 }
$$
$$
\textcolor{red}{2x_1 - 2 - x_1 = 6 \rightarrow x_1 = 8}
$$
\textcolor{red}{Portanto}
$$
\textcolor{red}{-8 + x_2 = 2 \rightarrow x_2 = 10}
$$

\textcolor{red}{O ponto $(8,10)$ é o ponto que maximiza a função, mas não queremos maximizá-la e sim minimizá-la. Como queremos minimizá-la é preciso ir na direção contrária a do ponto de maximização. Considerando as restrições e as condições de não negatividade, os únicos pontos que satisfazem o problema é o ponto de origem $(0,0)$ que faz parte da nossa região de solução. Conclui-se que os valor de mínimo dessa função é $0$.}

\begin{figure}[H]
    \centering
    \includegraphics[scale=.6]{Figuras/atv5/questao2b.PNG}
\end{figure}

\item[c)]

$$
\min f(x_1,x_2) = 3x_1 - 3x_2
$$

sujeito a 

$$
x_1 + x_2 \leq 6
$$
$$
x_1 - x_2 \leq 4
$$
$$
x_2 \leq 1
$$
$$
x_1 \geq 0; \quad x_2 \geq 0
$$

\noindent \textcolor{red}{Convertendo as inequações em equações:}

$$
\textcolor{red}{x_1 + x_2= 6}
$$
$$
\textcolor{red}{x_1 - x_2 = 4}
$$
$$
\textcolor{red}{x_1 = 1}
$$

\textcolor{red}{Para encontrarmos as retas das equações precisamos definir os pontos onde $x_1$ e $x_2$ são iguais a zero. }

\begin{itemize}
    \item \textcolor{red}{Para a primeira equação os pontos de sua reta são $A(0,6)$ e $B(6,0)$;}
    
    \item \textcolor{red}{Para a segunda equação os pontos de sua reta são $A(0,-4)$ e $B(4,0)$;}
    
     \item \textcolor{red}{Para a terceira equação $x_1 = 1$.}
\end{itemize}

 \textcolor{red}{Com base nessas informações obtemos o gráfico abaixo:}
 
 \begin{figure}[H]
    \centering
    \includegraphics[scale=.6]{Figuras/atv5/questao2c.PNG}
\end{figure}
 
 \textcolor{red}{A região em amarelo é onde, possivelmente, encontra-se a nossa solução. Para sabermos se a solução encontra-se dentro ou fora da região em amarelo, vamos aplicar o ponto $E(1,1)$ nas inequações e, se elas se mantiverem verdadeiras, então a nossa solução estará dentro dessa região.}
 
 $$
\textcolor{red}{x_1 + x_2 \leq 6 \rightarrow 1 + 1 \leq 6 \quad \rightarrow 2 \leq 6 \quad \textrm(verdadeira)}
$$
$$
\textcolor{red}{x_1 - x_2 \leq 4 \rightarrow 1 - 1 \leq 4 \quad \rightarrow 0 \leq 4 \quad \textrm(verdadeira)}
$$
$$
\textcolor{red}{x_1 \leq 1 \quad \textrm(verdadeira)}
$$

\textcolor{red}{Agora para encontrar a solução irei trabalhar com os múltiplos de 3 (já que a função objetivo é $3x_1 - 3x_2$), que serão 3,6 e 9. Entretanto, como o problema é de minimização, multiplica-se a função objetico por $-1$ e procuramos os pontos das retas:}

$$
\textcolor{red}{-3x_1 + 3x_2 = 3 \rightarrow M(0,1) \textrm{ e } N(-1,0)}
$$
$$
\textcolor{red}{-3x_1 + 3x_2 = 6 \rightarrow M(0,2) \textrm{ e } N(-2,0)}
$$
$$
\textcolor{red}{-3x_1 + 3x_2 = 9 \rightarrow M(0,3) \textrm{ e } N(-3,0)}
$$

 \begin{figure}[H]
    \centering
    \includegraphics[scale=.6]{Figuras/atv5/questao2c2.PNG}
\end{figure}

\textcolor{red}{Conforme o gráfico acima, a única reta que intercepta a região na cor amarela é a reta verde, da equação $-3x_1 + 3x_2 = 3$, cujo ponto é (0,1).}

\item[d)] Qual será a solução ótima se em c) decidirmos maximixar a função objetivo ao invés de minimizar. Esse problema terá mais de uma solução ótima? Justifique.

\textcolor{red}{Se decidirmos maximizar a função, a solução pode ser encontrada substituindo o valor de $x_1 = 1$ em $x_1 + x_2 = 6$, onde $x_2 = 5$. O ponto (5,1) é o que maximiza a função e o ponto (0,1) é o que minimiza-a.}

\end{itemize}
\end{section}

\newpage


\begin{section}{Questão 3}{Uma dada companhia produz vidro de alta qualidade, portas de alumínio e janelas. A empresa tem três pavilhões. No pavilhão 1 fabricam-se peças de alumínio, no pavilhão 2 peças de madeira e no 3º o vidro e a montagem final. A gestão considera a possibilidade de construir dois novos produtos. O produto A consiste numa nova porta de vidro com alumínio e o produto B uma janela com madeira. A gestão contratou um consultor para elaborar um estudo de viabilidade deste projeto. Segundo o levantamento do consultor o produto A consume 1 h por dia de trabalho no pavilhão 1 e 3h por dia no pavilhão 3 e retorna lucro de 3 mil reais. O produto B requer de 2 h por dia de trabalho no pavilhão 2 e 2 h por dia no pavilhão 3, produzindo um lucro esperado de 5 mil reais. Para não ser necessário contratar novos trabalhadores o pavilhão 1 só pode dispensar até 4 h por dia para fabricar os novos produtos, o pavilhão 2 pode ir até 12h por dia e o 3, 18 h por dia. Qual é o plano de produção proposto pelo engenheiro para maximizar o lucro?}

\begin{figure}[H]
    \centering
    \includegraphics[scale=0.95]{Figuras/atv5/q5.PNG}
\end{figure}

\noindent Dica: formule o problema como um modelo de programação linear e encontre a solução ótima pelo método gráfico. Usando o \textit{Cocalc}, confira se a solução obtida está correta.\\

\noindent \textcolor{red}{Construindo o problema: Considerando as informações temos como variáveis de decisão $x_1$ que é a quantidade produzida do produto A e $x_2$ que é a quantidade produzida do produto B e, respectivamente, esses produtos geram um lucro de 3 mil e 5 mil. Assim nossa função objetivo é}

$$
\textcolor{red}{L(x_1,x_2) = 3x_1 + 5x_2}
$$

\noindent \textcolor{red}{O objetivo é maximizar essa função objetivo, maximizar o lucro.}\\

\noindent \textcolor{red}{Dados que temos as função objetivo, agora iremos definir as restrições:}

 \begin{itemize}
     \item \textcolor{red}{Para o pavilhão 1 a capacidade disponível é de 4 h/dia e nesse pavilhão é produzida uma unidade do produto A, então essa primeira restrição será $1 \cdot x_1 \leq 4$ ou apenas $x_1 \leq 4$.}
     
     \item \textcolor{red}{Para o pavilhão 2 a capacidade disponível é de 12 h/dia e nesse pavilhão são produzidas duas unidades do produto B, então essa segunda restrição será $2 \cdot x_2 \leq 12$ ou apenas $2x_2 \leq 12$.}
     
     \item \textcolor{red}{E para terceiro pavilhão a capacidade disponível é de 18 h/dia e nesse pavilhão são produzidas três unidades do produto A e duas unidades quantidade do produto B, então essa terceira restrição será $3 \cdot x_1 + 2 \cdot x_2 \leq 18$ ou apenas $3x_1 + 2x_2 \leq 18$.}
 \end{itemize}
 
 
\noindent \textcolor{red}{Então temos como problema de otimização o seguinte:}
 
 $$
 \textcolor{red}{\min L(x_1,x_2) = 3x_1 + 5x_2}
$$

\textcolor{red}{sujeito a }

$$
\textcolor{red}{x_1 \leq 4}
$$
$$
\textcolor{red}{2x_2 \leq 12}
$$
$$
\textcolor{red}{3x_1 + 2x_2 \leq 18}
$$
$$
\textcolor{red}{x_1 \geq 0; \quad x_2 \geq 0}
$$

\noindent \textcolor{red}{Para a solução do problema, iremos converter as inequações em equações.}

$$
\textcolor{red}{x_1 = 4}
$$
$$
\textcolor{red}{2x_2 = 12 \rightarrow x_2 = 6}
$$
$$
\textcolor{red}{3x_1 + 2x_2 = 18}
$$

\noindent \textcolor{red}{É preciso obter a reta da terceira equação: quando $x_1 = 0$, então $x_2 = 9$ e quando $x_2 = 0$, então $x_1 = 6$. Portanto temos os pontos (0,9) e (6,0). Então, obtemos o seguinte gráfico:}

\begin{figure}[H]
    \centering
    \includegraphics[scale=0.7]{Figuras/atv5/questao3.PNG}
\end{figure}

\noindent \textcolor{red}{Para sabermos se a solução está dentro ou fora da região amarela substituiremos os valores de $x_1$ e $x_2$ pelos valores do ponto $E(1,1)$. Sendo os resultados verdadeiros, então a solução estará dentro da região amarela.}

$$
\textcolor{red}{x_1 \leq 4 \rightarrow 1 \leq 4 \quad \textrm(verdadeira)}
$$
$$
\textcolor{red}{2x_2 \leq 12 \rightarrow 2 \leq 12 \quad \textrm(verdadeira)}
$$
$$
\textcolor{red}{3x_1 + 2x_2 \leq 18 \rightarrow 3+2 \leq 18 \rightarrow 5 \leq 18 \quad \textrm(verdadeira)}
$$

\noindent \textcolor{red}{Agora irei testar possíveis soluções para a função objetivo de lucro. Sabemos que pelas equações $x_1 = 4$ e $x_2 = 6$ e se substituirmos cada um desses valores na terceira restrição teremos:}

$$
\textcolor{red}{x_1 = 4 \rightarrow 3 \cdot 4 + 2x_2 = 18 \rightarrow 12 + 2x_2 = 18 \rightarrow 2x_2 = 18 -12 \rightarrow 2x_2 = 6 \rightarrow x_2 = 3}
$$

\noindent \textcolor{red}{O ponto $(4,3)$ intercepta a reta da terceira restrição, conforme o gráfico.}

$$
\textcolor{red}{x_2 = 6 \rightarrow 3 \cdot 3x_1 + 2 \cdot 6 = 18 \rightarrow 3x_1 + 12 = 18 \rightarrow 3x_1 = 18 - 12 \rightarrow 3x_1 = 6 \rightarrow x_1 = 2}
$$

\noindent \textcolor{red}{O ponto $(2,6)$ intercepta a reta da terceira restrição, conforme o gráfico.}\\

\noindent \textcolor{red}{Então substituirei esses pontos encontrados na função objetivo de lucro e a que obtiver o maior valor a nossa solução.}

$$
\textcolor{red}{3 \cdot 4 + 5 \cdot 3 \rightarrow 12 + 15 = 27}
$$

$$
\textcolor{red}{3 \cdot 2 + 5 \cdot 6 \rightarrow 6 + 30 = 36}
$$

\noindent \textcolor{red}{Conforme resultados acima o ponto $(2,6)$ nos dá o maior lucro, logo essa será nossa solução. Conforme a imagem abaixo, o \textit{software Cocalc} encontrou o mesmo resultado.}

\begin{figure}[H]
    \centering
    \includegraphics[scale=0.8]{Figuras/atv5/questao32.PNG}
\end{figure}

\end{section}



















\end{document}

