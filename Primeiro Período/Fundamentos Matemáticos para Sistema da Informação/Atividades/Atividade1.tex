\documentclass[12pt]{article}
\usepackage[brazilian]{babel}
\usepackage[utf8]{inputenc}
\usepackage{graphicx}
\usepackage{placeins}
\usepackage{float}
\usepackage{rotate}              
\usepackage{enumerate, graphics} 
\usepackage{epsfig}
\usepackage{makeidx}
\usepackage[usenames,dvipsnames]{color}
\usepackage{fancyhdr}
\pagestyle{fancy}
\pagestyle{myheadings}
\fancyhf{}
\lhead{}
\rhead{\thepage}
\rfoot{}
\renewcommand{\headrulewidth}{0pt}


\usepackage{array}
\usepackage[justification=centering]{caption}
\usepackage{quoting}
\usepackage[singlelinecheck=false]{caption}
\usepackage{subcaption}
\usepackage{multirow}
\usepackage{multicol}
\usepackage{lmodern}
\usepackage[T1]{fontenc}
\newenvironment{dedication}
{%\clearpage           % we want a new page          %% I commented this
   \thispagestyle{empty}% no header and footer
   \vspace*{\stretch{1}}% some space at the top
   \itshape             % the text is in italics
   \raggedleft          % flush to the right margin
  }
  {\par % end the paragraph
   \vspace{\stretch{3}} % space at bottom is three times that at the top
   \clearpage           % finish off the page
  }
\usepackage{nomencl}
\makenomenclature
\usepackage{bigstrut}
\usepackage{booktabs}
\usepackage{longtable}
\usepackage{tabularx}
\usepackage{tabulary}
\usepackage{tabu}
\usepackage[toc,page]{appendix}
\usepackage{palatino}
\usepackage{amsmath,amsfonts,mathabx}
\usepackage{epigraph}
\usepackage{natbib}
\usepackage{geometry}
\usepackage{verbatim}
\usepackage{pifont}
\usepackage{setspace}
\usepackage{lscape}
\usepackage{pdflscape}
\usepackage{float}
\usepackage{amsmath}
\usepackage{nicefrac}
\usepackage[nottoc,notlof,notlot]{tocbibind}
\usepackage{booktabs,caption,fixltx2e} 
\usepackage[flushleft]{threeparttable}
\usepackage{tocloft}
\usepackage{times}
\renewcommand{\cftsecleader}{\cftdotfill{\cftdotsep}}


\geometry{top=3cm, bottom=2cm, right=2cm, left=3cm}



\begin{document}

\begin{titlepage}
\centering
 \vfill
  \begin{center}


   {\large {UNIVERSIDADE FEDERAL RURALDE PERNAMBUCO \\
BACHARELADO EM SISTEMAS DA INFORMAÇÃO
}}\\[4cm]

   {\large {\textbf{Aluno} \\
   Ivanildo Batista da Silva Júnior
   }}\\[.5cm]
   
   {\large {\textbf{Professora}\\ Dra. Silvana Bocanegra}}\\[3cm]
   {\large \textbf{Resolução da primeira lista de Fundamentos Matemáticos para Sistemas da Informação I}}\\[10.5cm] 
   
\normalsize {Recife-PE, \today}
\newpage

  \vfill
\end{center}
\end{titlepage}


\newpage

\tableofcontents
\thispagestyle{empty}
\newpage

\newpage
\setcounter{page}{1}

\begin{section}{Questão 1}{Aplique o algoritmo visto na aula 1 começando pela cidade C e responda:}

\begin{figure}[H]
    \centering
    \includegraphics[scale=1.15]{Figuras/questao1.PNG}
\end{figure}

\begin{itemize}
    \item[(a)] Qual á o subconjunto de arestas da malha rodoviária e qual a distância total percorrida.?
\end{itemize}

\noindent \textbf{\textcolor{red}{RESOLUÇÃO DA LETRA (A)}}\\

\noindent \textcolor{red}{Conforme a aula 1, partindo da cidade C, o subconjunto de arestas da rodoviária é $$\{\{C,B\},\{B,A\},\{A,D\},\{D,E\},\{E,F\}\}$$}

\noindent \textcolor{red}{De C para B temos 5, de B para A temos 5, de A para D temos 10, de D para E temos 20 e de E para F temos 10. Somando todas as distância de um vértice para o outro temos : $5 + 5 + 10 + 20 + 10 = 50$}

\begin{itemize}
    \item[(b)] Existe algum outro subconjunto de arestas desta malha rodoviária em que a distância obtida seja a mesma? Justifique
\end{itemize}

\noindent \textbf{\textcolor{red}{RESOLUÇÃO LETRA (B)}}\\

\noindent \textcolor{red}{Existem outros subconjuntos, mas dentre todas as possibilidades o outro subconjunto com a mesma distância do subconjunto solução da letra a).}\\

\noindent \textcolor{red}{Solução letra a): $\{\{C,B\},\{B,A\},\{A,D\},\{D,E\},\{E,F\}\} \rightarrow \boxed{5+5+10+20+10= 50}$}\\

\noindent \textcolor{red}{é o subconjunto:}\\

\noindent \textcolor{red}{$\{\{C,B\},\{B,A\},\{B,D\},\{D,E\},\{E,F\}\}
\rightarrow \boxed{5+5+10+20+10= 50}$ }\\


\begin{itemize}
    \item[(c)] Você acredita que aplicando esse algoritmo para um problema com 100 cidades na malha será possível garantir que a solução obtida é uma solução ótima (de menor distância)? Justifique.
\end{itemize}

\noindent \textbf{\textcolor{red}{RESOLUÇÃO LETRA (C)}}\\

\noindent\textcolor{red}{Sim, desde que os vértices do grafo gerado nesse problema não tenho um número ímpar de arestas, caso contrário não será possível ter um grafo euleriano.}








\end{section}
\newpage
%%%%%%%%%%%%%%%%%%%%%%%%%%%%%%%%%%%%%%%%%%%%%%%%%%%%%%%%%%%%%%%%%%%%%%%%%%%%%%%%%%%%%%%%%%%%%%%%%%%%%%%%%%%%%%%%%%%%%%%%%%%%%%%%%%%%%%%%%%%%%%%%%%%%%%%%%%%%%%%%%%%%%%%%%%%%%%%%%%%%%%%%%%%%%%%%%%%%%%%%%%%%%%%%%%%%%%%%%%%%%%%%
\begin{section}{Questão 2}{Faça um grafo representado as relações de amizade entre você e seus 05 melhores amigos no Facebook. Para cada um dos itens a seguir responda e justifique usando a definição.}\\

\noindent \textbf{\textcolor{red}{RESOLUÇÃO}}\\

\noindent \textcolor{red}{Abaixo temos o grafo com as arestas e vértices dos meus melhores amigos do Facebook.}\\

\begin{figure}[H]
    \centering
    \includegraphics[scale=1]{Figuras/questao22.PNG}
\end{figure}

\begin{itemize}
    \item[(a)] Qual é o vértice de maior grau ?\\
\noindent \textcolor{red}{O grau de um vértice é número de arestas que
incidem no vértice. O vértice de maior grau é o meu, que é de grau 5.}

    \item[(b)] Esse grafo é direcionado ?\\
\noindent \textcolor{red}{Para que o grafo seja direcionado as arestas devem ter direções. No caso do Facebook, a relação de amizade exige que um amigos aceite o convite de amizade, ninguém pode ser amigo sem que o outro aceite. Se fosse no Instagram, então podemos ter um grafo direcionado, pois é possível seguir uma pessoa e essa pessoa não seguir você. Logo, esse grafo não é direcionado.}
    
    \item[(c)] Esse grafo tem laços ?\\
\noindent \textcolor{red}{Não há laços, pois não é possível ter uma relação de amizade comigo mesmo.}
    
    \item[(d)] Tem arestas paralelas ?\\
\noindent \textcolor{red}{Não há arestas paralelas, pois não existem mais nenhuma outra relação de amizade com uma mesma pessoa.}
    
    \item[(e)] Tem ciclos ?\\
\noindent \textcolor{red}{Ciclos são trilhas fechadas e trilhas são passeios em que todas as arestas são distintas. O grafo com a relação de amizade possui ciclos, como por exemplo}

\textcolor{red}{$$\{\{\text{Denner},\text{Alyne}\},\{\text{Alyne},\text{Josué}\},\{\text{Josué},\text{Ivanildo}\},\{\text{Ivanildo},\text{Denner}\}\}$$}
    
    \item[(f)] É conexo ?\\
\textcolor{red}{Sim, pois não existe vértice sem conexão com algum outro vértice. Por via de regra, se todos são meus amigos, então não há desconexão entre os vértices.}
    
    \item[(g)] Qual a relação entre a soma do grau dos vértices e o número de arestas ?\\
\textcolor{red}{Teorema: A soma dos graus dos
vértices em um grafo é igual ao
dobro do número de arestas.}

\textcolor{red}{$$ \text{Número de graus dos vértice = 2 * Número de Arestas} $$}
    
\textcolor{red}{Para o nosso grafo: Graus dos vértices}

\textcolor{red}{$$\left \{ \begin{matrix}  d(Ivanildo) = 5\\ d(Josué) = 3 \\d(Denner) = 3 \\d(Alyne) = 3 \\d(Rafael) = 1\\ d(Henrique) = 1 \end{matrix} \right.$$}

\textcolor{red}{Somando os graus dos vértices temos o valor de 16. O número de arestas é igual a 8, conforme abaixo:}

\textcolor{red}{$$\{\{\text{I},\text{A}\},\{\text{I},\text{J}\},\{\text{I},\text{R}\},\{\text{I},\text{D}\},\{\text{I},\text{H}\},\{\text{J},\text{A}\}, \{\text{J},\text{D}\},\{\text{A},\text{D}\}\}$$}

\textcolor{red}{$$ \boxed{16 = 2 \cdot 8 }$$}

\end{itemize}

\end{section}
\newpage
%%%%%%%%%%%%%%%%%%%%%%%%%%%%%%%%%%%%%%%%%%%%%%%%%%%%%%%%%%%%%%%%%%%%%%%%%%%%%%%%%%%%%%%%%%%%%%%%%%%%%%%%%%%%%%%%%%%%%%%%%%%%%%%%%%%%%%%%%%%%%%%%%%%%%%%%%%%%%%%%%%%%%%%%%%%%%%%%%%%%%%%%%%%%%%%%%%%%%%%%%%%%%%%%%%%%%%%%%%%%%%%%
\begin{section}{Questão 3}{Sherlock Holmes foi acionado para desvendar um assassinato na residência de um bilionário cuja planta
da casa é apresentada na figura a seguir:\\

\noindent O mordomo alega ter visto o jardineiro entrar na sala da piscina (lugar onde ocorreu o assassinato) e logo em seguida viu-o sair daquela sala pela mesma porta que havia entrado.\\

\noindent O jardineiro afirma que ele não poderia ser a pessoa vista pelo mordomo, pois ele havia entrado na casa, passado por todas as portas uma única vez e, em seguida, deixado a casa.\\

\noindent Sherlock Holmes avaliou a planta da residência e em poucos minutos declarou solucionado o caso. Justifique qual foi o argumento usado pelo detetive para solucionar o caso.}

\begin{figure}[H]
    \centering
    \includegraphics[scale=1]{Figuras/questao2.PNG}
\end{figure}

(Dica: modele a planta da casa como um grafo e avalie a veracidade das afirmações feitas pelo jardineiro e pelo mordomo)
\newpage

\textbf{\textcolor{red}{RESOLUÇÃO}}\\

\textcolor{red}{Primeiro irei colocar o problema na forma de um grafo.}

\begin{figure}[H]
    \centering
    \includegraphics[scale=0.85]{Figuras/questao3.PNG}
\end{figure}

\noindent \textcolor{red}{Coloquei em destaque os vértices referentes a SALA PRINCIPAL e ao QUARTO PRINCIPAL. Vemos que esses vértices possuem um número ímpar de arestas o que impossibilita a existência de um circuito euleriano, por conta disso é impossível que exista um caminho nesse grafo em que cada vértice seja percorrido uma única vez. Sendo assim o JARDINEIRO ESTÁ MENTINDO ao afirmar que entrou na casa  e passou por todos os cômodos uma única vez e em seguida ter deixado a casa. \textbf{Logo o jardineiro é o assassino}.}

\end{section}
\newpage
%%%%%%%%%%%%%%%%%%%%%%%%%%%%%%%%%%%%%%%%%%%%%%%%%%%%%%%%%%%%%%%%%%%%%%%%%%%%%%%%%%%%%%%%%%%%%%%%%%%%%%%%%%%%%%%%%%%%%%%%%%%%%%%%%%%%%%%%%%%%%%%%%%%%%%%%%%%%%%%%%%%%%%%%%%%%%%%%%%%%%%%%%%%%%%%%%%%%%%%%%%%%%%%%%%%%%%%%%%%%%%%%
\begin{section}{Questão 4}{Considere o mapa abaixo mostrando quatro cidades ($A,B,C,D$) e as distância em $km$ entre elas. Determine
a menor distância a ser percorrida por um caixeiro viajante, considerando que ele deve sair da cidade $A$ visitar cada cidade exatamente uma vez e retornar a cidade $A$. (dica: enumere todos os circuitos
Hamiltonianos começando e terminando em $A$.}

\begin{figure}[H]
    \centering
    \includegraphics[scale=1]{Figuras/questao4.PNG}
\end{figure}

\textcolor{red}{RESOLUÇÃO}\\

\textcolor{red}{Para resolver essa questão preferi colocar as cidades conforme o diagrama abaixo:}


\begin{figure}[H]
    \centering
    \includegraphics[scale=.85]{Figuras/questao42.PNG}
\end{figure}

\noindent \textcolor{red}{Conforme o diagrama acima, os percursos que o Caixeiro Viajante pode percorrer na menor distância são $\{\{A,B\},\{B,C\},\{C,D\},\{D,A\}\} \text{ e } \{\{A,D\},\{D,C\},\{C,B\},\{B,A\}\}$, que possuem a mesma distância de 125 $Km$.}

\end{section}
\newpage

\begin{section}{Questão 5}{Considere o problema com 10 cidades. Qual o tempo necessário para resolver esse problema em um
computador equipado com um programa capaz de examinar 1 milhão de rotas por segundo ? E se agora tivéssemos que resolver o problema com 20 cidades. Ainda seria viável ?}\\

\noindent \textcolor{red}{Considerando um grafo completo para o problema, para calcular o número de soluções usa-se a fórmula:}

\textcolor{red}{$$(n-1)!$$}

\noindent \textcolor{red}{Sendo $n$ o número de vértices do grafo (as cidades) que temos 10 cidades, então}

\textcolor{red}{$$(10-1)! = 9! = 362880$$}

\noindent \textcolor{red}{Dividindo $362880$ por $1000000/s$, temos}

\textcolor{red}{$$\dfrac{\text{Número de rotas}}{\text{Velocidade de processamento do computador}} = \dfrac{362880}{1000000/s} = 0.36288 \text{ segundos}$$}

\noindent \textcolor{red}{Para um computador com velocidade de processamento 1000000 de rotas por segundo, o problema pode ser resolvido em menos de $4$ segundos.}\\

\noindent \textcolor{red}{Aumentando o probela para 20 cidades temos}

\textcolor{red}{$$(20-1)! = 19! = 1.216451 \cdot 10^{17}$$}

\noindent \textcolor{red}{Dividindo $1.216451 \cdot 10^{17}$ por $1000000/s$, temos}

\textcolor{red}{$$\dfrac{\text{Número de rotas}}{\text{Velocidade de processamento do computador}} = \dfrac{1.216451 \cdot 10^{17}}{1 \cdot 10^{6}/s} =  1.216451 \cdot 10^{11} \text{ segundos ou }$$}

\textcolor{red}{$$121645100409  \text{ segundos} $$}

\noindent \textcolor{red}{Irei converter esse resultado para horas dividindo por $3600$ que é o número de segundos de uma hora}

\textcolor{red}{$$\dfrac{121645100409}{3600} \approx  3379030 \text{ horas}$$}

\noindent \textcolor{red}{Irei converter esse resultado para dias dividindo por $24$ que é o número de horas de um dia}

\textcolor{red}{$$\dfrac{3379030}{24} \approx  1407929 \text{ dias}$$}

\noindent \textcolor{red}{Irei converter esse resultado para anos dividindo por $365$ que é o número de dias de um ano}

\textcolor{red}{$$\dfrac{1407929}{365} \approx 3857  \text{ anos}$$}

\noindent \textcolor{red}{O tempo para solucionar o problema com 20 cidades com um computador capaz de examinar 1 milhão de rotas por segundo é de 3857 anos, \textbf{logo esse problema é inviável de resolver}.}

\end{section}
\newpage
%%%%%%%%%%%%%%%%%%%%%%%%%%%%%%%%%%%%%%%%%%%%%%%%%%%%%%%%%%%%%%%%%%%%%%%%%%%%%%%%%%%%%%%%%%%%%%%%%%%%%%%%%%%%%%%%%%%%%%%%%%%%%%%%%%%%%%%%%%%%%%%%%%%%%%%%%%%%%%%%%%%%%%%%%%%%%%%%%%%%%%%%%%%%%%%%%%%%%%%%%%%%%%%%%%%%%%%%%%%%%%%%
\begin{section}{Questão 6}{Quantos vértices tem um grafo regular de grau 4 com 10 arestas.}\\

\noindent \textcolor{red}{Se o grafo é regular, então todos os vértices possuem o mesmo número de arestas, logo a soma dos graus é }

\noindent \textcolor{red}{$$ 4+4+4+\dots = 4n $$}

\noindent \textcolor{red}{A soma dos graus desse grafo é $4n$. Sabe-se também que a soma dos graus dos vértices em um grafo é igual ao dobro do número de arestas, então}

\noindent \textcolor{red}{$$ 4n = 2 \cdot 10 \therefore 4n = 20$$}

\noindent \textcolor{red}{Resolvendo}

\noindent \textcolor{red}{$$n = \dfrac{20}{4} = 5$$}

\noindent \textcolor{red}{$$\boxed{n = 5}$$}

\noindent \textcolor{red}{O número de vértices desse grafo é igual a 5.}

\end{section}
\newpage
%%%%%%%%%%%%%%%%%%%%%%%%%%%%%%%%%%%%%%%%%%%%%%%%%%%%%%%%%%%%%%%%%%%%%%%%%%%%%%%%%%%%%%%%%%%%%%%%%%%%%%%%%%%%%%%%%%%%%%%%%%%%%%%%%%%%%%%%%%%%%%%%%%%%%%%%%%%%%%%%%%%%%%%%%%%%%%%%%%%%%%%%%%%%%%%%%%%%%%%%%%%%%%%%%%%%%%%%%%%%%%%%
\begin{section}{Questão 7}{Verifique se os grafos abaixo sao eulerianos e/ou hamiltonianos. Exiba o circuito euleriano e/ou hamiltoniano caso seja possível}

\begin{figure}[H]
    \centering
    \includegraphics[scale=.85]{Figuras/questao7.PNG}
\end{figure}

\noindent \textbf{\textcolor{red}{RESOLUÇÃO A}}\\

\noindent \textcolor{red}{Para ser um grafo euleriano é necessário que esse grafo tenha um circuito euleriano , ou seja, que começa e termina no mesmo vértice e que passe por todas as arestas uma única vez, então eu posso passar por todos os vértices mais de uma vez. Como podemos ver abaixo o caminho}

\textcolor{red}{$$\{A, a1, B, a12, F, a10, G, a8, C, a3, D, a4, E, a9, G, a11, B, a2, C, a7, E, a5, F, a6, A \}$$}

\begin{figure}[H]
    \centering
    \includegraphics[scale=.65]{Figuras/caminho_euleriano_a.PNG}
\end{figure}

\noindent \textcolor{red}{Para o grafo ser hamiltoniano, esse grafo deve conter um caminho que passa por todos os vértices uma única vez e que começa e termina no mesmo vértice, então eu não preciso passar por todas as arestas do grafo. Assim sendo, conforme a seguir temos no grafo um circuito hamiltoniano.}\\

\textcolor{red}{$$\{A, a1, B, a11, G, a8, C, a3, D, a4, E, a5, F, a6, A \}$$}

\noindent \textcolor{red}{Outros caminhos aternativos alternativos}

\textcolor{red}{$$\{A, a6, F, a5, E, a4, D, a3, C,a8, G, a11, B, a1, A \}$$}

\textcolor{red}{$$\{A, a6, F, a10, G, a9, E, a4, D, a3, C, a2, B, a1, A \}$$}

\begin{figure}[H]
    \centering
    \includegraphics[scale=.65]{Figuras/caminho_hamiltoniano_a.PNG}
\end{figure}

\noindent \textcolor{red}{Como o grafo da letra A possui um circuito euleriano e um circuito hamiltoniano, então é um \textbf{grafo euleriano e hamiltoniano}.}\\

\noindent \textbf{\textcolor{red}{RESOLUÇÃO B}}\\

\noindent \textcolor{red}{Para que o grafo da letra B seja um grafo euleriano ele precisa ter uma circuito euleriano e, consequentemente, uma trilha euleriana. Sendo nosso grafo conexo, ele terá uma trilha euleriana se e somente se hoiver 0 ou 2 vértices de grau ímpar, entretanto nosso grafp possui 4 vértices de garu ímpar conforme a figura abaixo; portanto ele não é um grafo euleriano.}

\begin{figure}[H]
    \centering
    \includegraphics[scale=.65]{Figuras/caminho_euleriano_b.PNG}
\end{figure}

\noindent \textcolor{red}{Para que o grafo seja hamiltoniano, é preciso que exista um caminho que passe por todos os vértices uma única vez e termine o caminho no vértice de início. Conforme a imagem abaixo, existe um caminho que atende essas condições, portanto ele é um grafo hamiltoniano.}

\begin{figure}[H]
    \centering
    \includegraphics[scale=.65]{Figuras/caminho_hamiltoniano_b.PNG}
\end{figure}

\noindent \textcolor{red}{O caminho da imagem é:}

\textcolor{red}{$$\{A, a1, B, a2, C, a3, F, a4, I, a5, H, a6, G, a7, D, a15, E, a9, A \}$$}

\noindent \textcolor{red}{Mas também pode ser o caminho:}

\textcolor{red}{$$\{A, a9, E, a15, D, a7, G, a6, H, a5, I, a4, F, a3, C, a2, B, a1, A\}$$}

\noindent \textcolor{red}{Portanto o grafo da letra B não é euleriano, mas é hamiltoniano.}

\end{section}
\newpage

\end{document}

