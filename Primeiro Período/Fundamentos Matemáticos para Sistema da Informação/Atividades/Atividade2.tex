\documentclass[12pt]{article}
\usepackage[brazilian]{babel}
\usepackage[utf8]{inputenc}
\usepackage{graphicx}
\usepackage{placeins}
\usepackage{float}
\usepackage{amsmath}
\usepackage{blkarray}
\usepackage[svgnames]{xcolor}
\usepackage{rotate}              
\usepackage{enumerate, graphics} 
\usepackage{epsfig}
\usepackage{makeidx}
%\usepackage{tikz}
\usepackage[usenames,dvipsnames]{color}
\usepackage{fancyhdr}
\pagestyle{fancy}
\pagestyle{myheadings}
\fancyhf{}
\lhead{}
\rhead{\thepage}
\rfoot{}
\renewcommand{\headrulewidth}{0pt}


\usepackage{array}
\usepackage[justification=centering]{caption}
\usepackage{quoting}
\usepackage[singlelinecheck=false]{caption}
\usepackage{subcaption}
\usepackage{multirow}
\usepackage{multicol}
\usepackage{lmodern}
\usepackage[T1]{fontenc}
\newenvironment{dedication}
{%\clearpage           % we want a new page          %% I commented this
   \thispagestyle{empty}% no header and footer
   \vspace*{\stretch{1}}% some space at the top
   \itshape             % the text is in italics
   \raggedleft          % flush to the right margin
  }
  {\par % end the paragraph
   \vspace{\stretch{3}} % space at bottom is three times that at the top
   \clearpage           % finish off the page
  }
\usepackage{nomencl}
\makenomenclature
\usepackage{bigstrut}
\usepackage{booktabs}
\usepackage{longtable}
\usepackage{tabularx}
\usepackage{tabulary}
\usepackage{tabu}
\usepackage[toc,page]{appendix}
\usepackage{palatino}
\usepackage{amsmath,amsfonts,mathabx}
\usepackage{epigraph}
\usepackage{natbib}
\usepackage{geometry}
\usepackage{verbatim}
\usepackage{pifont}
\usepackage{setspace}
\usepackage{lscape}
\usepackage{pdflscape}
\usepackage{float}
\usepackage{amsmath}
\usepackage{nicefrac}
\usepackage[nottoc,notlof,notlot]{tocbibind}
\usepackage{booktabs,caption,fixltx2e} 
\usepackage[flushleft]{threeparttable}
\usepackage{tocloft}
\usepackage{times}
\renewcommand{\cftsecleader}{\cftdotfill{\cftdotsep}}


\geometry{top=3cm, bottom=2cm, right=2cm, left=3cm}



\begin{document}

\begin{titlepage}
\centering
 \vfill
  \begin{center}


   {\large {UNIVERSIDADE FEDERAL RURAL DE PERNAMBUCO \\
BACHARELADO EM SISTEMAS DA INFORMAÇÃO
}}\\[4cm]

   {\large {\textbf{Aluno} \\
   Ivanildo Batista da Silva Júnior
   }}\\[.5cm]
   
   {\large {\textbf{Professora}\\ Dra. Silvana Bocanegra}}\\[3cm]
   {\large \textbf{Resolução da segunda lista de Fundamentos Matemáticos para Sistemas da Informação I}}\\[10.5cm] 
   
\normalsize {Recife-PE, \today}
\newpage

  \vfill
\end{center}
\end{titlepage}


\newpage

\tableofcontents
\thispagestyle{empty}
\newpage

\newpage
\setcounter{page}{1}

\begin{section}{Questão 1}{Em um canil de reabilitação de cães abandonados, a equipe acabou de receber 6 novos cachorros encontrados, e precisa colocá-los em diferentes casinhas para passar a primeira noite. Constatou-se que alguns
desses cães ficam extremamente agressivos na presença de alguns do mesmo grupo, logo, terão que alocá-los em casinhas diferentes. O cachorro A não pode ficar com os cachorros C, D ou E. Os cachorros B e
F não podem ficar juntos. O cachorro E não pode ficar com os cachorros D nem F. Quantas casinhas serão necessárias para acomodar os 6 cachorros? (dica: formule o problema como um grafo e determine o número cromático).}\\

\noindent \textcolor{red}{Para visualizar melhor o enunciado, irei colocar em forma de tabela para organizar as informações.}

\begin{table}[H]
\centering
\begin{tabular}{ccc}
\hline
\textcolor{red}{Cachorro} & \textcolor{red}{Não pode ficar junto} & Pode\\ \hline
\textcolor{red}{A} & \textcolor{red}{C, D ou E} & \textcolor{red}{B ou F}\\
\textcolor{red}{B} &\textcolor{red}{F} & \textcolor{red}{A, C, D e/ou E}\\
\textcolor{red}{F} &\textcolor{red}{B} & \textcolor{red}{A, C, D e/ou E}\\
\textcolor{red}{E} & \textcolor{red}{D e/ou F} & \textcolor{red}{A, B e/ou C}\\ \hline
\end{tabular}%
\end{table}

\noindent \textcolor{red}{Agora irei colocar em forma de grafo.}

\begin{figure}[H]
    \centering
    \includegraphics[scale=.85]{Figuras/atv2/Q2r1.PNG}
\end{figure}

\noindent \textcolor{red}{Conforme o grafo acima, temos o valor 3 como número cromático do nosso grafo, Portanto o número de casinhas necessárias para abrigar esse 6 cachorros é igual a 3.}


\end{section}

\newpage
%%%%%%%%%%%%%%%%%%%%%%%%%%%%%%%%%%%%%%%%%%%%%%%%%%%%%%%%%%%%%%%%%%%%%%%%%%%%%%%%%%%%%%%%%%%%%%%%%%%%%%%%%%%%%%%%%%%%%%%%%%%%%%%%%%%%%%%%%%%%%%%%%%%%%%%%%%%%%%%%%%%%%%%%%%%%%%%%%%%%%%%%%%%%%%%%%%%%%%%%%%%%%%%%%%%%%%%%%%%%%%%%
\begin{section}{Questão 2}{Determine o número cromático dos grafos a seguir}\\

\begin{figure}[H]
    \centering
    \includegraphics[scale=.85]{Figuras/atv2/Q2.PNG}
\end{figure}

\noindent \textcolor{red}{\underline{Para o grafo $G_1$}:}

\begin{figure}[H]
    \centering
    \includegraphics[scale=.65]{Figuras/atv2/Q2r2.PNG}
\end{figure}

\noindent \textcolor{red}{O grafo $G_1$ tem o valor 3 como número cromático, pois o número mínimo de cores necessárias para a sua coloração é 3 (vermelha, verde e azul).}

\newpage

\noindent \textcolor{red}{ \underline{Para o grafo $G_2$}:}

\begin{figure}[H]
    \centering
    \includegraphics[scale=.65]{Figuras/atv2/q2r3.PNG}
\end{figure}

\noindent \textcolor{red}{O grafo $G_2$ tem o valor 3 como número cromático, pois o número mínimo de cores necessárias para a sua coloração é 3 (vermelha, verde e azul), assim como no grafo $G_1$.}\\

\noindent \textcolor{red}{\underline{Para o grafo $G_3$}:}

\begin{figure}[H]
    \centering
    \includegraphics[scale=.65]{Figuras/atv2/q2r4.PNG}
\end{figure}

\noindent \textcolor{red}{O grafo $G_3$ tem o valor 4 como número cromático, pois o número mínimo de cores necessárias para a sua coloração é 4 (vermelha, verde, azul e amarelo), assim como no grafo $G_1$.}

\newpage

\noindent \textcolor{red}{\underline{Para o grafo $G_4$}:}

\begin{figure}[H]
    \centering
    \includegraphics[scale=.65]{Figuras/atv2/q2r5.PNG}
\end{figure}

\noindent \textcolor{red}{O grafo $G_4$ tem o valor 2 como número cromático, pois o número mínimo de cores necessárias para a sua coloração é  (vermelha e verde).}

\end{section}
\newpage

\begin{section}{Questão 3}{Determine quais dos grafos do exercício 2 são planares.}\\


\noindent \textcolor{red}{\underline{Para o grafo $G_1$}: Não há cruzamento entre as arestas, portanto ele é um grafo planar.}\\

\noindent \textcolor{red}{\underline{Para o grafo $G_2$}: A imagem abaixo mostra o grafo $G_2$ redesenhado de uma forma em que suas arestas não se cruzam, logo ele é um grafo planar.}

\begin{figure}[H]
    \centering
    \includegraphics[scale=.65]{Figuras/atv2/q3r2.PNG}
\end{figure}

\noindent \textcolor{red}{\underline{Para o grafo $G_3$}: Conforme o grafo abaixo, não é possível encontrar outro caminho para a aresta destacada na cor vermelha em que não ocorra um cruzamento com outras arestas, por esse motivo esse grafo não é planar.}

\begin{figure}[H]
    \centering
    \includegraphics[scale=.65]{Figuras/atv2/q3r3.PNG}
\end{figure}

\newpage 

\noindent \textcolor{red}{\underline{Para o grafo $G_4$}: Abaixo temos o grafo $G_4$ redesenhado e a aresta destacada na forma tracejada não consegue chegar no vértice no meio da parte de baixo sem que ocorra um cruzamento; por esse motivo esse grafo não é planar.}

\begin{figure}[H]
    \centering
    \includegraphics[scale=.65]{Figuras/atv2/q3r4.PNG}
\end{figure}

\end{section}
\newpage

\begin{section}{Questão 4}{Emissoras de televisão vão ser instaladas em estações em oito cidades de nosso estado. Segundo regulamento, a mesma emissora não pode ser instalada em duas cidades com distância inferior a 150Km. As distâncias entre as cidades estão descritas na tabela abaixo. Indique o menor número de emissoras que
devem ser instaladas para contemplar as nove cidades?}\\

\begin{figure}[H]
    \centering
    \includegraphics[scale=.85]{Figuras/atv2/Q4.PNG}
\end{figure}

\noindent \textcolor{red}{Para solucionar esse problema, irei considerar as cidades onde pode ter a instalação de estações, conforme a tabela abaixo}

\begin{table}[H]
\centering
\begin{tabular}{l|cccccccc}
\hline
\multicolumn{1}{c|}{} & \multicolumn{1}{l}{A} & \multicolumn{1}{l}{B} & \multicolumn{1}{l}{C} & \multicolumn{1}{l}{D} & \multicolumn{1}{l}{E} & \multicolumn{1}{l}{F} & \multicolumn{1}{l}{G} & H \\ \hline
\multicolumn{1}{c|}{A} & - &  &  &  & X &  & X &  \\
\multicolumn{1}{c|}{B} &  & - & X &  &  &  & X &  \\
C &  &  & - & X &  & X & X &  \\
D &  &  &  & - & X &  & X &  \\
E &  &  &  &  & - & X & X &  \\
F &  &  &  &  &  & - &  & X \\
G &  &  &  &  &  &  & - & X \\
\multicolumn{1}{c|}{H} &  &  &  &  &  &  &  & - \\ \hline
\end{tabular}%
\end{table}

\noindent \textcolor{red}{Com base nessas informações criei o grafo a seguir:}

\begin{figure}[H]
    \centering
    \includegraphics[scale=.65]{Figuras/atv2/q4r1.PNG}
\end{figure}

\noindent \textcolor{red}{O número mínimo de cores que usei nesse grafo foram 4, assim são necessárias 4 estações para serem instaladas. Outra informação importante é que, se formos colocar essas localizações em um mapa e coloríssemos ele, teríamos, segundo Francis Guthrie 4 cores para essa coloração. Conforme prova de Appel e Haken, em 1977, todo mapa no plano pode ser colorido com no máximo 4 cores.}


\end{section}
\newpage

\begin{section}{Questão 5}{Considere o grafo a seguir:}

 \begin{figure}[H]
    \centering
    \includegraphics[scale=.85]{Figuras/atv2/Q5.PNG}
\end{figure}

\begin{itemize}
    \item[(a)] Verifique se o grafo é bipartido.
    
 \begin{figure}[H]
    \centering
    \includegraphics[scale=.65]{Figuras/atv2/q5r1.PNG}
\end{figure}

\noindent \textcolor{red}{Conforme (I) na imagem a acima o nosso grafo possui número mínimo de cores igual a 2. Em (II) o grafo foi colocado em uma forma intermediária e em (III) trocamos as posições dos vértices para que estejam do mesmo lado, assim temos um grafo com dois grupos distintos. Conclui-se que o grafo é bipartido.}
    
    \item[(b)] Esse grafo contém ciclo (circuito) ímpar ?

\noindent \textcolor{red}{Conforme a imagem abaixo, a resposta é não, pois o ciclo é par (possui um número par de vértices).}
    
 \begin{figure}[H]
    \centering
    \includegraphics[scale=.65]{Figuras/atv2/q5r2.PNG}
\end{figure}

    
    \item[(c)] Os vértices desse grafo podem ser particionados em dois subconjuntos de vertices independentes?

\noindent \textcolor{red}{Só pelo fato de ser um grafo bipartido já seria possível separar em dois grupos independentes. Na imagem abaixo vemos os dois grupos desse grafos.}


 \begin{figure}[H]
    \centering
    \includegraphics[scale=.65]{Figuras/atv2/q5r3.PNG}
\end{figure}

\end{itemize}


\end{section}
\newpage

\begin{section}{Questão 6}{Represente os grafos a seguir usando lista encadeada e matriz de adjacência}\\

\begin{figure}[H]
    \centering
    \includegraphics[scale=.85]{Figuras/atv2/Q6.PNG}
\end{figure}

\noindent \textcolor{red}{\underline{Para o grafo $G_1$}: lista encadeada}

$$ \textcolor{red}{Adj[V_1] = [v_4,v_5,v_6]}$$
$$ \textcolor{red}{Adj[V_2] = [v_4,v_5,v_6]}$$
$$ \textcolor{red}{Adj[V_3] = [v_4,v_5,v_6]}$$
$$ \textcolor{red}{Adj[V_4] = [v_1,v_2,v_3]}$$
$$ \textcolor{red}{Adj[V_5] = [v_1,v_2,v_3]}$$
$$ \textcolor{red}{Adj[V_6] = [v_1,v_2,v_3]}$$

\noindent \textcolor{red}{Matriz de adjancência para o grafo $G_1$:}

\[ \mathbf{A}= \begin{blockarray}{r*{6}{ >{\color{black}}c}}
& v_1 & v_2 & v_3 & v_4 & v_5 & v_6\\
\begin{block}{ >{\scriptstyle}r!{\,}(cccccc)}
v_1 & 0 & 0 & 0 & 1 & 1 & 1 \\
v_2 & 0 & 0 & 0 & 1 & 1 & 1 \\
v_3 & 0 & 0 & 0 & 1 & 1 & 1\\
v_4 & 1 & 1 & 1 & 0 & 0 & 0\\
v_5 & 1 & 1 & 1 & 0 & 0 & 0 \\
v_6 & 1 & 1 & 1 & 0 & 0 & 0 \\
\end{block}
\end{blockarray} \]

 \newpage
 
\noindent \textcolor{red}{\underline{Para o grafo $G_2$}: lista de adjacência.}
 
$$ \textcolor{red}{Adj[r] = [u,v,x,y,z]}$$
$$ \textcolor{red}{Adj[u] = [r,v]}$$
$$ \textcolor{red}{Adj[v] = [u,w,y,z]}$$
$$ \textcolor{red}{Adj[w] = [r,v,x,y,z]}$$
$$ \textcolor{red}{Adj[x] = [r,w,y]}$$
$$ \textcolor{red}{Adj[y] = [r,v,w,x,z]}$$
$$ \textcolor{red}{Adj[z] = [r,v,w,y]}$$

\noindent \textcolor{red}{Matriz de adjacência para o grafo $G_2$:}
\[ \mathbf{A}= \begin{blockarray}{r*{7}{ >{\color{black}}c}}
& r & u & v & w & x & y & z\\
\begin{block}{ >{\scriptstyle}r!{\,}(ccccccc)}
r & 0 & 1 & 1 & 1 & 1 & 1 & 1 \\
u & 1 & 0 & 1 & 0 & 0 & 0 & 0 \\
v & 0 & 1 & 0 & 1 & 0 & 1 & 1\\
w & 1 & 0 & 1 & 0 & 1 & 1 & 1\\
x & 1 & 0 & 0 & 1 & 0 & 1 & 0 \\
y & 1 & 0 & 1 & 1 & 1 & 0 & 1 \\
z & 1 & 0 & 1 & 1 & 0 & 1 & 0\\
\end{block}
\end{blockarray} \]

\end{section}
\newpage

\begin{section}{Questão 7}{Aplique o algoritmo do vizinho mais próximo para resolver o problema do caixeiro viajante no grafo a seguir e responda.}

\begin{figure}[H]
    \centering
    \includegraphics[scale=.85]{Figuras/atv2/Q7.PNG}
\end{figure}

%\tikzstyle{level 1}=[level distance=1.5cm, %s%ibling distance=2.5cm]
%\tikzstyle{level 2}=[level distance=1.25cm, sibling distance=.75cm]
%\tikzstyle{level 3}=[level distance=1cm, sibling distance=.5cm]

% Define styles for bags and leafs
%\tikzstyle{bag} = [text width=1em]
%\tikzstyle{end} = [circle, minimum width=5pt,fill, inner sep=1pt]

%\begin{tikzpicture}[grow=right, sloped]
%\node[bag] {a}
 %   child {node[bag] {b} 
  %      child {node[bag]{c}
   %     child {node[bag]{d} child {node[bag]{e}}}
    %    child {node[bag]{e}child {node[bag]{d}}}}
     %   child {node[bag]{d}}
      %5  child {node[bag]{e}}
      % 5 }
    %child {node[bag] {d}
     %   child {node[bag]{b}}
      %  child {node[bag]{c}}
       % child {node[bag]{e}}
        %}
%\end{tikzpicture}

\begin{itemize}
    \item[(a)] Qual a distância percorrida obtida usando essa heurística ?\\
    
\noindent \textcolor{red}{Utilizando os 5 vértices do grafo:}

$$\text{\textcolor{red}{Partindo de a : }} [a,b,e,d,c] \rightarrow 3+2+1+6 = 12$$
$$\text{\textcolor{red}{Partindo de b : }} [b,e,d,a,c] \rightarrow 2+1+4+8 = 15$$
$$\text{\textcolor{red}{Partindo de c : }} [c,e,d,a,b] \rightarrow 5+1+4+3 = 13$$
$$\text{\textcolor{red}{Partindo de d : }} [d,e,b,a,c] \rightarrow 1+2+3+8 = 14$$
$$\text{\textcolor{red}{Partindo de e : }} [e,d,a,b,c] \rightarrow 1+2+3+10 = 16$$\\

\textcolor{red}{Utilizando a heurística do Vizinho mais Próximo, a menor distância percorrido é a que inicia do vértice \textit{a} com um valor de 12.}
    
    \item[(b)] Compare com o valor obtido usando um algoritmo de força bruta.
    
\noindent \textcolor{red}{Utilizando o algoritmo de força bruta, é preciso identificar todas as possibilidades e testá-las uma a uma. Para 5 vértices temos 24 possibilidades de caminho partindo de um vértice aleatório. São 24 combinações segundo a fórmula $(n-1)!$, logo $(5-1)! = 4! = 24$. A seguir irei usar todos os caminhos possíveis para esse grafo partindo do vértice \textit{a}.}


$$[a,b,c,d,e] \rightarrow 3+10+6+1 = 20$$
$$[a,b,c,e,d] \rightarrow 3+10+5+1 = 19$$
$$[a,b,d,c,e] \rightarrow 3+9+6+5 = 23$$
$$[a,b,d,e,c] \rightarrow 3+9+1+5 = 18$$
$$[a,b,e,c,d] \rightarrow 3+2+5+6 = 16$$
$$\boxed{[a,b,e,d,c] \rightarrow 3+2+1+6 = 12}$$

$$[a,c,b,d,e] \rightarrow 8+10+9+1 = 28$$
$$[a,c,b,e,d] \rightarrow 8+10+2+1 = 21$$
$$[a,c,d,b,e] \rightarrow 8+6+9+2 = 25$$
$$[a,c,d,e,b] \rightarrow 8+6+1+2 = 17$$
$$[a,c,e,b,d] \rightarrow 8+5+2+9 = 16$$
$$[a,c,e,d,b] \rightarrow 8+5+1+9 = 23$$

$$[a,d,b,c,e] \rightarrow 4+9+10+5 = 28$$
$$[a,d,b,e,c] \rightarrow 4+9+2+5 = 20$$
$$[a,d,c,b,e] \rightarrow 4+6+10+2 = 22$$
$$[a,d,c,e,b] \rightarrow 4+6+5+2 = 17$$
$$[a,d,e,b,c] \rightarrow 4+1+2+10 = 17$$
$$[a,d,e,c,b] \rightarrow 4+1+5+10 = 20$$

$$[a,e,b,c,d] \rightarrow 7+2+10+6 = 25$$
$$[a,e,b,d,c] \rightarrow 7+2+9+6 = 24$$
$$[a,e,c,b,d] \rightarrow 7+5+10+9 = 31$$
$$[a,e,c,d,b] \rightarrow 7+5+6+9 = 27$$
$$[a,e,d,b,c] \rightarrow 7+1+9+10 = 27$$
$$[a,e,d,c,b] \rightarrow 7+1+6+10 = 24$$

\noindent \textcolor{red}{Tanto na heurística de Vizinhos mais próximos quando no algoritmo de Força Bruta, encontramos um caminho para o caixeiro viajante cujo valor é 12. Entretanto, com a heurística de Vizinhos mais próximos realizamos os cálculos com 6 caminhos diferentes, enquanto no algoritmo de Força Bruta o cálculo foi feito com 24 caminhos diferentes.}
    
    \item[(c)] Em que circunstâncias devemos escolher uma heurística?
    
\noindent \textcolor{red}{Quando não for viável utilizar o algoritmo de Força Bruta. Por exemplo, se esse grafo tivesse 6 vértices teríamos 6 soluções difernetes na heurística de Vizinhos mais Próximos, mas no algoritmo de Força Bruta teríamos 120 soluções diferentes para avaliar, o que é inviável.}

\end{itemize}


\end{section}
\newpage
\end{document}

