\documentclass[12pt]{article}
\usepackage[brazilian]{babel}
\usepackage[utf8]{inputenc}
\usepackage{graphicx}
\usepackage{placeins}
\usepackage{float}
\usepackage{amsmath}
\usepackage{blkarray}
\usepackage[svgnames]{xcolor}
\usepackage{amsmath}
\usepackage{rotate}              
\usepackage{enumerate, graphics} 
\usepackage{epsfig}
\usepackage{makeidx}
%\usepackage{tikz}
\usepackage[usenames,dvipsnames]{color}
\usepackage{fancyhdr}
\pagestyle{fancy}
\pagestyle{myheadings}
\fancyhf{}
\lhead{}
\rhead{\thepage}
\rfoot{}
\renewcommand{\headrulewidth}{0pt}


\usepackage{array}
\usepackage[justification=centering]{caption}
\usepackage{quoting}
\usepackage[singlelinecheck=false]{caption}
\usepackage{subcaption}
\usepackage{multirow}
\usepackage{multicol}
\usepackage{lmodern}
\usepackage[T1]{fontenc}
\newenvironment{dedication}
{%\clearpage           % we want a new page          %% I commented this
   \thispagestyle{empty}% no header and footer
   \vspace*{\stretch{1}}% some space at the top
   \itshape             % the text is in italics
   \raggedleft          % flush to the right margin
  }
  {\par % end the paragraph
   \vspace{\stretch{3}} % space at bottom is three times that at the top
   \clearpage           % finish off the page
  }
\usepackage{nomencl}
\makenomenclature
\usepackage{bigstrut}
\usepackage{booktabs}
\usepackage{longtable}
\usepackage{tabularx}
\usepackage{tabulary}
\usepackage{tabu}
\usepackage[toc,page]{appendix}
\usepackage{palatino}
\usepackage{amsmath,amsfonts,mathabx}
\usepackage{epigraph}
\usepackage{natbib}
\usepackage{geometry}
\usepackage{verbatim}
\usepackage{pifont}
\usepackage{setspace}
\usepackage{lscape}
\usepackage{pdflscape}
\usepackage{float}
\usepackage{amsmath}
\usepackage{nicefrac}
\usepackage[nottoc,notlof,notlot]{tocbibind}
\usepackage{booktabs,caption,fixltx2e} 
\usepackage[flushleft]{threeparttable}
\usepackage{tocloft}
\usepackage{times}
\renewcommand{\cftsecleader}{\cftdotfill{\cftdotsep}}


\geometry{top=3cm, bottom=2cm, right=2cm, left=3cm}



\begin{document}

\begin{titlepage}
\centering
 \vfill
  \begin{center}


   {\large {UNIVERSIDADE FEDERAL RURAL DE PERNAMBUCO \\
BACHARELADO EM SISTEMAS DA INFORMAÇÃO
}}\\[4cm]

   {\large {\textbf{Aluno} \\
   Ivanildo Batista da Silva Júnior
   }}\\[.5cm]
   
   {\large {\textbf{Professora}\\ Dra. Silvana Bocanegra}}\\[3cm]
   {\large \textbf{Resolução da quarta lista de Fundamentos Matemáticos para Sistemas da Informação I}}\\[10.5cm] 
   
\normalsize {Recife-PE, \today}
\newpage

  \vfill
\end{center}
\end{titlepage}


\newpage

\tableofcontents
\thispagestyle{empty}
\newpage

\newpage
\setcounter{page}{1}

\begin{section}{Questão 1}{Uma empresa possui três fábricas, A, B e C que produzem 100, 120 e 120 toneladas de um determinado produto, respectivamente. O produto deverá ser entregue em cinco armazéns (1, 2, 3, 4 e 5), cada um dos quais deve receber 40, 50, 70, 90 e 90 toneladas, respectivamente. Os custos, por tonelada, de transporte entre cada fábrica e cada armazém são dados na tabela seguinte. Formule um modelo de programação linear para minimizar os custos de transporte}

\begin{figure}[H]
    \centering
    \includegraphics[scale=1]{Figuras/atv4/q1.PNG}
\end{figure}

\noindent \textcolor{red}{RESOLUÇÃO:}\\

\noindent \textcolor{red}{\underline{Variáveis de decisão}: Quantidade produzida pela fábrica A ($X_1$), Quantidade produzida pela fábrica B ($X_2$) e Quantidade produzida pela fábrica C ($X_3$). Essas quantidades serão separadas para os armazéns 1, 2, 3, 4 e 5, portanto irei separar essas quantidades das seguntes formas:}

\begin{itemize}
    \item \textcolor{red}{Quantidades de A para os armazéns : $x_{11},x_{12},x_{13},x_{14} \textrm{ e } x_{15}$}.
    
    \item \textcolor{red}{Quantidades de B para os armazéns : $x_{21},x_{22},x_{23},x_{24} \textrm{ e } x_{25}$}.
    
    \item \textcolor{red}{Quantidades de C para os armazéns : $x_{31},x_{32},x_{33},x_{34} \textrm{ e } x_{35}$}.
\end{itemize}

\noindent \textcolor{red}{\underline{Função objetivo:} O objetivo é minimizar os custos, portanto é preciso encontrar a função custo (C) que é a soma dos produtos das quantidades de cada armazém e os seus respectivos custos:}

$$
    C = 4x_{11} + x_{12} + 2x_{13} + 6x_{14} + 9x_{15} + 6x_{21} + 4x_{22} + 3x_{23} + 5x_{24} + 7x_{25} + 5x_{31} + 2x_{32} + 6x_{33} + 4x_{34} + 8x_{35}
$$

\noindent \textcolor{red}{\underline{Restrições técnicas:}}\\

\noindent $x_{11} + x_{12} + x_{13} + x_{14} + x_{15}  \leq 100$\\
$x_{21} + x_{22} + x_{23} +x_{24} + x_{25}  \leq 120$\\
$x_{31} + x_{32} + x_{33} + x_{34} + x_{35} \leq 120$\\
$x_{11} + x_{21} + x_{31} \leq 40$\\
$x_{12} + x_{22} + x_{32} \leq 50$\\
$x_{13} + x_{23} + x_{33} \leq 70$\\
$x_{14} + x_{24} + x_{34} \leq 90$\\
$x_{15} + x_{25} + x_{35} \leq 90$\\
$x_{11} \geq 0, \quad x_{12} \geq 0, \quad x_{13} \geq 0, \quad x_{14} \geq 0, \quad x_{15} \geq 0$\\
$x_{21} \geq 0, \quad x_{22} \geq 0, \quad x_{23} \geq 0, \quad x_{24} \geq 0, \quad x_{25} \geq 0$\\
$x_{31} \geq 0, \quad x_{32} \geq 0, \quad x_{33} \geq 0, \quad x_{34} \geq 0, \quad x_{35} \geq 0$

\end{section}
\newpage

\begin{section}{Questão 2}{Uma empresa automobilística produz dois tipos de carros (modelo A e modelo B). Cada um destes carros pode ser fabricado em duas oficinais. A oficina 1 tem um máximo de 120 horas de trabalho disponível e a oficina 2 um máximo de 180 h. A fabricação do carro do modelo A requer 6 horas de trabalho na oficina 1 e 3 h na oficina 2. A fabricação do modelo B requer 4 h na oficina 1 e 1 hora na oficina 2. O lucro é de 30 mil no carro A e de 40 mil no carro B. Formule um modelo de programação linear que maximize o lucro na produção dos carros.}\\

\noindent \textcolor{red}{RESOLUÇÃO:}\\

\noindent \textcolor{red}{\underline{Variáveis de decisão}: Quantidade produzida pela empresa para o carro A ($X_1$) e a quantidade produzida pela empresa para o carro B ($X_2$). Essas quantidades serão fabricadas por duas oficinas (1 e 2). As quantidade produzidas pela oficina 1 serão $x_{11}$ e $x_{12}$; já as quantidades produzidas pela oficina 2 serão $x_{21}$ e $x_{22}$.}\\

\noindent \textcolor{red}{\underline{Função objetivo:} O objetivo é maximizar o lucro, portanto é preciso encontrar a função lucro (L) que é a soma dos produtos das quantidades produzidas para cada tipo de carro e do lucro por unidade de cada tipo de carro:}

$$
    L = 30000X_{1} + 40000X_{2}
$$

\noindent \textcolor{red}{\underline{Restrições técnicas:}}\\

\noindent $6x_{11} + 4x_{21} \leq 120$\\
$3x_{12} + 4x_{22} \leq 180$\\
$x_{11} \geq 0, \quad x_{12} \geq 0 \quad x_{21} \geq 0, \quad x_{22} \geq 0$

\end{section}
\newpage


\begin{section}{Questão 3}{Um criador de gado pretende determinar as quantidades de cada tipo de ração que devem ser
dadas diariamente a cada animal por forma a reduzir os custos com a ração e garantir as restrições nutricionais. Os dados relativos ao custo de cada tipo de ração, às quantidades mínimas diárias de ingredientes nutritivos básicos a fornecer a cada animal, bem como às quantidades destes existentes em cada tipo de ração (g/Kg), constam do quadro a seguir. Formule um modelo de programação linear que minimize os custos com a ração}

\begin{figure}[H]
    \centering
    \includegraphics[scale=1]{Figuras/atv4/q3.PNG}
\end{figure}

\noindent \textcolor{red}{RESOLUÇÃO:}\\

\noindent \textcolor{red}{\underline{Variáveis de decisão}: A variáveis de decisão são a quantidade para ração granulada é $X_1$ e a quantidade de ração do tipo farinha é $X_2$.}\\

\noindent \textcolor{red}{\underline{Função objetivo:} O objetivo é minimizar o custo, portanto é preciso encontrar a função custo total (C) que é a soma dos produtos da quantidade dadas ao gado e o custo por tipo de ração. Prtanto a função custo é:}

$$
    C = 10X_{1} + 5X_{2}
$$

\noindent \textcolor{red}{\underline{Restrições técnicas:}}\\

\noindent $20X_{1} + 50x_{2} \geq 200$\\
$50X_{1} + 50X_{2} \leq 150$\\
$30X_{1} + 30X_{2} \leq 210$\\
$X_{1} \geq 0 \quad \text{e} \quad X_{2} \geq 0$

\end{section}
\newpage

\begin{section}{Questão 4}{Um gerente de investimentos de um banco gerencia recursos de terceiros através da escolha de carteiras de investimento. Para sugerir a carteira de um de seus clientes, precisa levar em consideração a análise de risco efetuada, a qual determina que:}

\begin{itemize}
    \item[-] Não mais de 25\% do total seja aplicado em um único investimento.
    
    \item[-] Mais de 50\% do total deve ser aplicado em títulos de maturidade de mais de 10 anos.
    
    \item[-] O total aplicado em títulos de alto risco deve ser no máximo de 50\% do total investido. 
\end{itemize}


\begin{figure}[H]
    \centering
    \includegraphics[scale=1]{Figuras/atv4/q4.PNG}
\end{figure}

\end{section}

\noindent \textcolor{red}{RESOLUÇÃO:}\\

\noindent \textcolor{red}{\underline{Variáveis de decisão}: Para o ativo com retorno de $8.7\%$ a quantidade é $X_1$; para o ativo com retorno anual de $9.5\%$ a quantidade é dada por $X_2$; a quantidade do ativo com retorno de $12\%$ é dada por $X_3$; a quantidade do ativo com retorno anual de $9\%$ é $X_4$; a quantidade do ativo com retorno anual de $13\%$ é $X_5$; e por fim, a quantidade do ativo com retorno anual de $20\%$ é $X_6$.}\\

\noindent \textcolor{red}{\underline{Função objetivo:} O objetivo é maximizar o retorno da carteira de investimentos, portanto é preciso encontrar a função retorno (R) que é a soma dos produtos da quantidade de cada investimento pela taxa de retorno anual de cada ativo. Portanto a função retorno total é:}

$$
    R = 0.087X_{1} + 0.095X_{2} + 0.12X_{3} + 0.09X_{4} + 0.13X_{5} + 0.2X_{6}
$$

\noindent \textcolor{red}{\underline{Restrições técnicas:}}\\

\noindent $X_{1} + X_{2} + X_5 \geq 0.5$\\
$X_{3} + X_{4} + X_6 \leq 0.5$\\
$X_{1} + X_{2} +  X_{3} + X_{4} + X_5 + X_6 \leq 1$\\
$X_{1} \geq 0, X_{2} \geq 0, X_{3} \geq 0, X_{4} \geq 0, X_{5} \geq 0, X_{6} \geq 0$










\newpage
\end{document}

